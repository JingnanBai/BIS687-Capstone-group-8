\documentclass[12pt,]{article}
\usepackage{lmodern}
\usepackage{amssymb,amsmath}
\usepackage{ifxetex,ifluatex}
\usepackage{fixltx2e} % provides \textsubscript
\ifnum 0\ifxetex 1\fi\ifluatex 1\fi=0 % if pdftex
  \usepackage[T1]{fontenc}
  \usepackage[utf8]{inputenc}
\else % if luatex or xelatex
  \ifxetex
    \usepackage{mathspec}
  \else
    \usepackage{fontspec}
  \fi
  \defaultfontfeatures{Ligatures=TeX,Scale=MatchLowercase}
    \setmainfont[]{Arial}
\fi
% use upquote if available, for straight quotes in verbatim environments
\IfFileExists{upquote.sty}{\usepackage{upquote}}{}
% use microtype if available
\IfFileExists{microtype.sty}{%
\usepackage{microtype}
\UseMicrotypeSet[protrusion]{basicmath} % disable protrusion for tt fonts
}{}
\usepackage[margin=.5in]{geometry}
\usepackage{hyperref}
\hypersetup{unicode=true,
            pdftitle={Group 8 Proposal},
            pdfauthor={Jingnan Bai, Kexin Wang, Yiming Miao, Qiyu Huang},
            pdfborder={0 0 0},
            breaklinks=true}
\urlstyle{same}  % don't use monospace font for urls
\usepackage{natbib}
\bibliographystyle{plainnat}
\IfFileExists{parskip.sty}{%
\usepackage{parskip}
\setlength{\parskip}{8pt plus 2pt minus 1pt}
}{% else
\setlength{\parindent}{0pt}
\setlength{\parskip}{8pt plus 2pt minus 1pt}
}
\setlength{\emergencystretch}{3em}  % prevent overfull lines
\providecommand{\tightlist}{%
  \setlength{\itemsep}{0pt}\setlength{\parskip}{0pt}}
\setcounter{secnumdepth}{5}

%%% Use protect on footnotes to avoid problems with footnotes in titles
\let\rmarkdownfootnote\footnote%
\def\footnote{\protect\rmarkdownfootnote}


  \title{Group 8 Proposal}
    \author{Jingnan Bai, Kexin Wang, Yiming Miao, Qiyu Huang}
      \date{2024-02-20}


%%%%%%%%%%
% personal preamble edits here
%%%%%%%%%%
\pagenumbering{gobble}

% can toggle this for Helvetica
%\usepackage{helvet}
%\renewcommand{\familydefault}{\sfdefault}

% \titlespacing*{\paragraph}{0pt}{2pt}{1em}

% set section numbering/lettering
% tips for seccntformat: https://tex.stackexchange.com/questions/95896/how-to-format-subsection-title-without-packages
\makeatletter
\def\@seccntformat#1{%
  \expandafter\ifx\csname c@#1\endcsname\c@section\else
  \expandafter\ifx\csname c@#1\endcsname\c@paragraph\else
  \csname the#1\endcsname\quad
  \fi\fi}
  
  % stucture for these commands: https://texfaq.org/FAQ-atsigns
  \renewcommand\section{
  \@startsection{section}{1}{\z@}
    {-3.5ex \@plus -1ex \@minus -.2ex}
    {1.0ex \@plus.2ex} %reduce space below section (was 1.5ex)
    {\normalfont\normalsize\bf\uppercase}} %modify font style
    
  \renewcommand\subsection{
  \@startsection{subsection}{2}{\z@}
    {-1.5ex\@plus -1ex \@minus -.2ex}%reduce space above subsection (was -3.25ex)
    {0.5ex \@plus .2ex}%reduce space below subsection (was 1.5ex)
    {\normalfont\normalsize\bf}} %modify font style
    
  \renewcommand\subsubsection{
  \@startsection{subsubsection}{3}{\z@}
    {-1.0ex\@plus -1ex \@minus -.2ex}%reduce space above subsubsection (was -3.25ex)
    {0.5ex \@plus .2ex}%reduce space below subsubsection (was 1.5ex)
    {\normalfont\normalsize\bf}} %modify font style
    
  \renewcommand\paragraph{
  \@startsection{paragraph}{4}{\z@}
    {-0.5ex\@plus -1ex \@minus -.2ex}%reduce space above paragraph (was -3.25ex)
    {-1.5ex \@plus .2ex}%convert space below paragraph to an indent (was 1.5ex)
    {\normalfont\normalsize\bf}} %modify font style    
\makeatother

\renewcommand\thesubsection{\Alph{subsection}.}
\renewcommand\thesubsubsection{\thesubsection\arabic{subsubsection}.}

% reduce spacing at the top of lists
\usepackage{enumitem}
\setlist{topsep = 2pt}

% allow text to wrap around figures
\usepackage{graphicx}
\usepackage{wrapfig}

%%%%%%%%%%

\begin{document}
\maketitle

\hypertarget{specfic-aims}{%
\section{Specfic Aims}\label{specfic-aims}}

Parkinson's disease (PD) ranks as the second-most common
neurodegenerative disorder in the US, affecting nearly 90,000
individuals annually. Characterized by uncontrollable movements such as
shaking, stiffness, and imbalance, PD significantly diminishes patients'
quality of life. Defined as a progressive neurological disorder, PD
occurs due to the gradual degeneration or death of nerve cells in the
basal ganglia. The exact cause of neuronal death remains inconclusive,
though current research suggests a combination of specific genetic
variants and environmental factors, including toxin exposure, as leading
risk contributors.

The UK Biobank is a large biomedical database that contains detailed
genetic, clinical, and lifestyle information on 500,000 UK participants.
Notably, it includes precise records of PD diagnoses among participants,
offering a unique lens to longitudinally examine the characteristics of
PD patients and hence deepen our understanding of the disease's onset.

This research project, informed by a thorough review of the dataset and
existing medical literature, is driven by two specific aims:

\hypertarget{specific-aim-1}{%
\subsection{Specific Aim 1}\label{specific-aim-1}}

To analyze the influence of patient information on PD occurrence.
Leveraging the UK Biobank's extensive data on demographics, habits,
diets, and environmental exposures, we aim to construct a statistical
model that explains the relationship between these variables and PD
occurrence, identifying factors with significant inference.

\hypertarget{specific-aim-2}{%
\subsection{Specific Aim 2}\label{specific-aim-2}}

To develop predictive models for early PD onset. Beyond demographic and
lifestyle data, the UK Biobank provides in-depth details on
participants' limbs and trunks. Utilizing all these data, our objective
is to create a predictive model with high accuracy to help the early
detection of PD.

We anticipate that this project will uncover factors that elucidate or
predict the onset of PD. We hope such findings could enable individuals
at higher risk to benefit from earlier interventions and contribute to
the well-being among aging people.

\hypertarget{research-strategy}{%
\section{Research Strategy}\label{research-strategy}}

\hypertarget{significance}{%
\subsection{Significance}\label{significance}}

Analyzing Parkinson's disease (PD) data is crucial for advancing our
grasp of this intricate neurodegenerative condition, enhancing how we
diagnose and treat it. Through the examination of PD-related
information, researchers can identify not-so-obvious patterns, including
genetic, environmental, and lifestyle factors that influence the
disease's development. This insight is instrumental in creating models
to detect individuals at increased risk sooner, facilitating early
actions that can profoundly modify the course of PD. Moreover, this
research aids in pinpointing biomarkers for the early identification and
tracking of PD, essential for customizing treatment plans. It also
propels the creation of innovative treatments by revealing how PD
operates at a fundamental level. In essence, delving into PD data not
only promises to elevate the life quality of those affected globally but
also kindles hope for more effective management and the pursuit of a
cure.

\hypertarget{innovation}{%
\subsection{Innovation}\label{innovation}}

We face a significant challenge in our research on Parkinson's disease
(PD) because of the naturally skewed dataset, which has fewer PD
patients than the non-affected population. This imbalance might skew our
analysis and produce less trustworthy findings. Our study uses methods
intended to efficiently balance the dataset in order to address this
problem in a novel way. Techniques like undersampling the control group,
oversampling PD cases, or using synthetic data generation, such as SMOTE
(Synthetic Minority Over-sampling Technique), are used. By normalizing
the dataset distribution, these techniques help to improve the accuracy
of our analysis and make it more representative of the true impact of
PD. Our goal is to improve the basis for future PD research by
proactively addressing this imbalance and gaining insights that are both
applicable and valid.

Regarding model selection, our approach transcends traditional methods
by embracing a variety of predictive models, including Logistic
Regression, Lasso Regression, and Ridge Regression, each known for its
unique strengths in handling specific types of data characteristics. To
bolster the robustness and accuracy of our predictions, we also
integrate advanced ensemble techniques like BOOSTING and BAGGING. These
methods, by pooling predictions from multiple models, significantly
enhance our analytical framework's performance. They help in reducing
the risk of overfitting and increasing model reliability. This
comprehensive and innovative modeling strategy not only elevates the
precision of our PD predictions but also deepens our understanding of
the disease's intricate patterns, offering a solid basis for developing
more effective diagnostic and therapeutic interventions.

\hypertarget{research-plan}{%
\subsection{Research Plan}\label{research-plan}}

We present our research on Parkinson's disease (PD) in four condensed
steps for maximum efficiency and clarity. The first step of the process
is data processing, where we deal with the imbalance in our dataset,
which is a typical problem in medical research. We guarantee fair
representation between PD cases and controls by using methods like SMOTE
for oversampling and strategic undersampling, providing a foundation
free of bias for our analysis.

Transitioning to the exploratory data analysis (EDA) phase, our goal is
to sift through the data for preliminary insights and identify potential
PD predictors. This critical step not only helps us grasp the dataset's
underlying structure but also steers the direction of our deeper
investigation.

We use complex algorithms in the feature importance analysis that
follows to identify the variables that have the biggest impact on the
risk of Parkinson's disease. This stage is critical because it reveals
the relative importance of each feature and their causal relationships,
deepening our comprehension of the complex nature of PD. The
understanding we gain from this experience will be crucial to
interpreting our results in a meaningful and persuasive way.

Our work culminates in the model building stage, where we use multiple
predictive techniques, such as Ridge, Lasso, and Logistic Regression, as
well as ensemble approaches like BOOSTING and BAGGING. Our models are
put through a rigorous evaluation process to make sure they have strong
generalization capabilities in addition to being an accurate predictor
of PD risk. This rigorous and concise methodology ensures a targeted
investigation of PD data, with the goal of providing deep insights into
its forecast and broad management approaches.

\hypertarget{specific-aim-1-to-investigate-the-combined-effects-of-patient-demographics-dietary-patterns-and-lifestyle-habits-on-the-occurrence-of-parkinsons-disease.}{%
\subsection{Specific Aim 1: To investigate the combined effects of
patient demographics, dietary patterns, and lifestyle habits on the
occurrence of Parkinson's
disease.}\label{specific-aim-1-to-investigate-the-combined-effects-of-patient-demographics-dietary-patterns-and-lifestyle-habits-on-the-occurrence-of-parkinsons-disease.}}

\hypertarget{hypothesis}{%
\subsubsection{Hypothesis}\label{hypothesis}}

Patient demographics (age, gender, ethnicity), diet (specific dietary
patterns or nutritional intake), and habits (exercise frequency, tobacco
use, alcohol intake) can significantly affect the occurrence of
Parkinson's disease.

\hypertarget{rationale}{%
\subsubsection{Rationale}\label{rationale}}

The emphasis on these factors is due to the multifactorial nature of
Parkinson's disease, which involves complex interactions between genetic
predisposition and environmental exposures. In previous clinical
research, it has been established that specific dietary components such
as antioxidants found in fruits and vegetables can reduce oxidative
stress, which is crucial in the neurodegenerative process observed in
Parkinson's disease. While these results provide valuable information on
potential protective dietary elements, a key question remains in our
understanding of how broader lifestyle patterns, including exercise,
tobacco use, and alcohol consumption, interact with dietary habits and
demographic backgrounds to influence the risk of Parkinson's disease.

\hypertarget{experimental-approach}{%
\subsubsection{Experimental Approach}\label{experimental-approach}}

\begin{itemize}
\item
  Data Preparation: Missing data will be handled based on the nature and
  extent of the missingness. Dietary intake and exercise data will be
  converted into standardized units such as servings per day, and hours
  per week to enable consistent comparisons. Additionally, we may also
  normalize variables with skewed distributions to satisfy the
  assumptions of certain statistical tests.
\item
  Descriptive Statistics: Summary statistics (mean, median, standard
  deviation, frequencies) will be calculated to provide a comprehensive
  overview of the data and identify potential issues for further
  investigation.
\item
  Data Visualization: We will employ individual plots or heatmaps for
  interaction effects to visualize the relationship between factors and
  the outcome.
\item
  Statistical Model: Recognizing the challenge of multicollinearity,
  where closely related factors may interface with the ability to
  discern their individual effects, we will first select potential
  factors in each category. We may consider using models that include
  penalties such as LASSO or ridge regression. Subsequent to feature
  selection, we will conduct statistical tests to assess the
  significance of these factors. As longitudinal measurements for each
  feature are available, we will implement mixed-effect models to
  capture the within-participant variability over time and provide an
  understanding of how individual trajectories influence the outcome.
\item
  Sensitivity Analysis: We will perform sensitivity analyses to validate
  model assumptions and ensure our findings regarding key factors are
  reliable.
\end{itemize}

\hypertarget{interpretation-of-results}{%
\subsubsection{Interpretation of
Results}\label{interpretation-of-results}}

Given the results from the statistical models, we will interpret the
results in the context of existing literature, and how our results align
with or diverge from previous studies. We will also discuss the
practical implications for the prevention and management of Parkinson's
disease.

\hypertarget{potential-problems-and-alternative-approaches}{%
\subsubsection{Potential Problems and Alternative
Approaches}\label{potential-problems-and-alternative-approaches}}

Unaddressed confounding variables, which influence both the exposure
(diet, lifestyle habits) and the outcome (Parkinson's disease), may bias
the results. Socioeconomic status, for example, may impact participants'
access to healthcare resources and safe environments for exercise, which
could affect the occurrence of the disease. A potential solution
involves adjusting for other known confounders in the multivariate
model.

\hypertarget{specific-aim-2-establishing-practical-and-interpretable-models-for-predicting-the-risk-of-parkinsons-disease.}{%
\subsection{Specific Aim 2: Establishing practical and interpretable
models for predicting the risk of Parkinson's
disease.}\label{specific-aim-2-establishing-practical-and-interpretable-models-for-predicting-the-risk-of-parkinsons-disease.}}

\hypertarget{hypothesis-1}{%
\subsubsection{Hypothesis}\label{hypothesis-1}}

Parkinson's disease is widely believed to be associated with a
combination of various risk factors, making the prediction of the
disease's risk feasible. Based on previous detection and analysis, we
will utilize interpretable predictive models to further validate
identified Parkinson's-related features, while exploring more about
indicators that may not be causative but could assist in early
Parkinson's diagnosis.

\hypertarget{rationale-1}{%
\subsubsection{Rationale}\label{rationale-1}}

While no definitive cause has been identified in previous studies,
various features have been linked to the risk of Parkinson's disease,
including genetics, environmental risk factors, lifestyle, and family
history of related syndromes. Given the potential interaction among
these factors, early diagnosis of Parkinson's disease could be complex
and challenging, which denotes that more comprehensive methods are
required to help model complex scenarios, capture relationships, and
achieve more accurate risk predictions.

\hypertarget{experimental-approach-1}{%
\subsubsection{Experimental Approach}\label{experimental-approach-1}}

To construct a practical risk prediction model, we will conduct data
processing, model training, and evaluation based on the following steps:

\begin{itemize}
\item
  Data Preparation: Involving data cleaning, checking for missing data,
  and basic preprocessing to ensure data quality. Considering the large
  number of features in the dataset, we will utilize data compression
  and other efficient feature filtering methods to reduce model
  complexity.
\item
  Feature Engineering: Exploring the relationship between features and
  the response variable with exploratory data analysis and basic
  visualization tools, and conducting appropriate feature engineering
  such as binning, encoding and spline tricks if needed. Additionally,
  considering the data imbalance, we will focus on sampling techniques
  and related evaluation in data processing to assist the model learning
  process.
\item
  Model Training and Improvement: Utilizing various ensemble learning
  methods (boosting or bagging) to construct predictive models for
  estimating the risk of Parkinson's disease. We will further improve
  model performance with appropriate parameter tuning and regularization
  strategies, considering the potential problem of overfitting with
  imbalanced data and sampling preprocessing.
\item
  Model Evaluation: With cross-validation techniques, further assessing
  the predictive performance of the models based on basic metrics for
  classification tasks, along with AUC for evaluating the model's
  ranking abilities. Furthermore, given the data characteristics, more
  attention will be attached to examining the robustness of the models
  when facing the imbalanced data in practical use.
\item
  Interpretation: Conduct model selection comprehensively based on
  multiple metrics. Calculating feature importance using ensemble
  learning methods to complement and support the conclusion of causal
  analysis in the previous section. Also, analyze indicators that may
  not have causality but could assist in Parkinson's disease diagnosis,
  providing insights into its early detection.
\end{itemize}

\hypertarget{interpretation-of-results-1}{%
\subsubsection{Interpretation of
Results}\label{interpretation-of-results-1}}

The predictive model will focus on a balance between accuracy and
interpretability, which would provide a new perspective on examining the
risk factors of Parkinson's disease. With the analysis of feature
importance and the information value of each feature, we will further
identify features that have causal relationships with Parkinson's
disease risk or could assist in early detection of Parkinson's disease.

\hypertarget{potential-problems-and-alternative-approaches-1}{%
\subsubsection{Potential Problems and Alternative
Approaches}\label{potential-problems-and-alternative-approaches-1}}

Given the extreme class imbalance and the complex interactions among
risk factors of Parkinson's disease, oversampling techniques like SMOTE
may introduce ambiguity to classification boundaries and exacerbate
potential confusion and overfitting. To address the possible challenges,
we plan to use cost-sensitive learning algorithms and other
methodologies as alternative approaches to mitigate these issues.

\hypertarget{reference}{%
\section{Reference}\label{reference}}

Chairta PP, Hadjisavvas A, Georgiou AN, et al.~Prediction of Parkinson's
Disease Risk Based on Genetic Profile and Established Risk Factors.
Genes (Basel). 2021;12(8):1278. Published 2021 Aug 20.
\url{doi:10.3390/genes12081278}

Yoon SY, Park YH, Lee HJ, Kang DR, Kim YW. Lifestyle Factors and
Parkinson Disease Risk: Korean Nationwide Cohort Study With Repeated
Health Screening Data. Neurology. 2022;98(6):e641-e652.
\url{doi:10.1212/WNL.0000000000012942}

\bibliography{references.bib}


\end{document}
